\documentclass{article}
\usepackage{amsmath} % for equations
\usepackage{graphicx} % to insert images

\title{Maths for Donut.c}
\author{Chahyn Ettaghi}
\date{September 2025}

\begin{document}

\maketitle

\section{Introduction}
Before anything I would advise checking \underline{https://www.a1k0n.net/2011/07/20/donut-math.html} to understand better this article.

Let's note the rotation matrices useful for the following: 
% Rotation around the Y axis
\[
R_y(\phi) =
\begin{pmatrix}
\cos\phi & 0 & \sin\phi \\
0 & 1 & 0 \\
-\sin\phi & 0 & \cos\phi
\end{pmatrix}
\]

% Rotation around the X axis
\[
R_x(A) =
\begin{pmatrix}
1 & 0 & 0 \\
0 & \cos A & \sin A \\
0 & -\sin A & \cos A
\end{pmatrix}
\]

% Rotation around the Z axis
\[
R_z(B) =
\begin{pmatrix}
\cos B & \sin B & 0 \\
-\sin B & \cos B & 0 \\
0 & 0 & 1
\end{pmatrix}
\]

A torus is defined by two radii and two rotation angles: 

\[
R_2 = \text{distance from the tube center to the torus center}, \quad
R_1 = \text{tube radius}.
\]

A point of the torus in 3D coordinates (before rotations via A and B) is given by: 

\[
(x,y,z) =
\bigl((R_2 + R_1 \cos \theta)\cos \phi, R_1 \sin \theta,\; 
(R_2 + R_1 \cos \theta)\sin \phi \; 
 \bigr).
\]

where:  

- $\theta$ traverses the circular section of the tube,  
- $\varphi$ goes around the torus.  

\medskip

Then, to animate it, we apply two successive rotations: around the $z$ axis (angle $B$) and rotation around the $x$ axis (angle $A$).  

\medskip

The final position of a point of the torus after 3D transformation is obtained by applying these two rotations to $(x,y,z)$ (in the following order): \[
R_y(\phi)R_x(A)R_z (B )
\]

\[
\begin{bmatrix}
x1 \\
y1 \\
z1
\end{bmatrix}
=
\begin{pmatrix}
(R_2 + R_1 \cos\theta)\cos\phi \cos B + \big(R_1 \sin\theta \cos A + (R_2 + R_1 \cos\theta)\sin\phi \sin A\big)\sin B \\[1mm]
-(R_2 + R_1 \cos\theta)\cos\phi \sin B + \big(R_1 \sin\theta \cos A + (R_2 + R_1 \cos\theta)\sin\phi \sin A\big)\cos B \\[1mm]
(R_2 + R_1 \cos\theta)\sin\phi \cos A - R_1 \sin\theta \sin A
\end{pmatrix}
\]


\section*{Rigorous demonstration of the torus normal}

We are going to determine the normal to the torus.

\subsection*{1. Torus parametrization}

As said before, a torus is parameterized by two angles: 
Without rotation, the 3D coordinates of a point of the torus are:
\[
r(\theta, \phi) = 
\big( (R_2 + R_1 \cos\theta)\cos\phi,R_1 \sin\theta,\; (R_2 + R_1 \cos\theta)\sin\phi\;  \big)
\]

\subsection*{2. Tangent vectors of the surface}

The surface is parameterized in \((\theta, \phi)\).  
Its tangent vectors are:
\[
\frac{\partial \vec{r}}{\partial \theta} =
\begin{pmatrix}
- R_1 \sin\theta \cos\phi \\
R_1 \cos\theta\\
- R_1 \sin\theta \sin\phi 

\end{pmatrix},
\qquad
 \frac{\partial \vec{r}}{\partial \phi} =
\begin{pmatrix}
- (R_2 + R_1 \cos\theta) \sin\phi \\
0\\
(R_2 + R_1 \cos\theta) \cos\phi 
\end{pmatrix}.
\]


\subsection*{3. Normal = cross product}

The normal is given by:
\[
\vec{N} = \frac{\partial \vec{r}}{\partial \theta} \times \frac{\partial \vec{r}}{\partial \phi}
\]


Which is:
\[
\vec{N} = R_1(R_2 + R_1 \cos\theta) 
\big( \cos\theta \cos\phi,\; \sin\theta,\; \cos\theta \sin\phi \big)
\]


The normal vectors are proportional to \(\vec{N}\), that is:
\[
\big( \cos\theta \cos\phi,\; \sin\theta,\; \cos\theta \sin\phi \big) \in \mathrm{Vect}(\vec{N})
\]




\section*{Dot product of the normal with light}


The normal vector after torus parametrization (before rotation) is therefore:
\[
\tilde{\mathbf{N}}_0 = \big( \cos\theta \cos\phi,\; \sin\theta,\; \cos\theta \sin\phi \big)
\]

Then after rotation: 

\[
\tilde{\mathbf{N}} = \tilde{\mathbf{N}}_0 \, R_z(B)R_x(A)
\]


\[
\tilde{\mathbf{N}} =
\begin{pmatrix} 
\cos\theta \cos\phi \cos B - \sin\theta \sin B \\ 
\cos A(\cos\theta \cos\phi \sin B + \sin\theta \cos B) - \cos\theta \sin\phi \sin A \\ 
\sin A(\cos\theta \cos\phi \sin B + \sin\theta \cos B) + \cos\theta \sin\phi \cos A 
\end{pmatrix}
\]

The light is directed by:
\[
\vec{L} = (0, 1, -1)
\]

The dot product gives the luminance \(Lu\):
\[
\begin{aligned}
Lu &= \tilde{\mathbf{N}} \cdot \vec{L} \\
&= (\cos A - \sin A)(\cos\theta \cos\phi \sin B + \sin\theta \cos B) - \cos\theta \sin\phi (\cos A + \sin A)
\end{aligned}
\]



\end{document}